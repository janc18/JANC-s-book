\documentclass{report} 
\usepackage[english]{babel}
\usepackage[utf8]{inputenc}

\usepackage{listings}
\usepackage{xcolor}
\lstset{
  backgroundcolor=\color{gray!10}, % Color de fondo
  frame=single, % Marco alrededor del código
  rulecolor=\color{black}, % Color del marco
  basicstyle=\ttfamily, % Estilo de fuente
  keywordstyle=\color{blue}, % Color para palabras clave
  numbers=left, % Números de línea a la izquierda
  numberstyle=\tiny, % Estilo de los números de línea
  breaklines=true, % Saltar líneas largas
  language=TeX % Lenguaje (LaTeX en este caso)
}

\title{LaTeX Basic commands}
\author{José Agustín}
\date{\today}

\begin{document}

\maketitle 

\tableofcontents 

\chapter{Introduction}
This document is about the LaTeX basic commands and how to use them in my particular case.
This is not a guide of how to use \LaTeX.
\\
\textbf{What will be my particular case?}
\\
Well, I want to use LaTeX to write some mathematical equations(which is one of its principal purposes),
Music anotations,to have a really good platform to do a litte changes quikly and with the conbination of vim
make my owm shortcuts,for example to create tables, lists and equations.

\chapter{Most used commands}
Here I will obviously put all the commands that I consider more important.
\enumerate
\item Basic document structure
\item Tables
\item Lists
\item Mathematicals equations
\item Section and subsections
\item Custom scripts to generate elements quikly

\section{Basic document structure}

\begin{lstlisting}
\documentclass{article}
\begin{document}

\end{document}
\end{lstlisting}

\textbf{documentclass} determines the document type and define its general format.
\\
Here are the list of the different types of formats available

\itemize
\item article: For creation of shorts documents and articles, like: informs,technical documents
	this format include: unique page with sections, subsections and summary.
\item report: For creation of long documents like: tesis, long informs and extended academic works,
	it give aditionally features like: numeric chapters.
\item book: For creation of books and manuscripts. Give a structure for chapters, sections,
	hierarchy subsections and allow the inclusion of preface and table of contents.
\item letter: For creations of letters, give specific formats for headers, adresses and signature
\item beamer: For creations of slide presentations
\\
\textbf{begin and end}


\section{Tables}
\section{Lists}
\section{Mathematicals equations}
\section{Section and subsections}
\section{Custom scripts to generate elements quikly}
%\begin{lstlisting}
%\begin{verbatim}
%\enumerate
%\item Primer elemento
%\item Segundo elemento
%\end{verbatim}    
%\end{lstlisting}

\chapter{How to organize the information}
Here I will add how to effciently organize the information.

\end{document}
  
