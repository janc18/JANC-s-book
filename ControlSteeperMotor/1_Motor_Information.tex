\documentclass{report} 
\usepackage[english]{babel}
\usepackage[utf8]{inputenc}
\usepackage{gensymb}
\usepackage{tcolorbox}

\newenvironment{deductionR}{
  \begin{tcolorbox}[colback=orange!10, colframe=orange!40!black, title=Deduction result]
}{
  \end{tcolorbox}
}

\newenvironment{doubt}{
  \begin{tcolorbox}[colback=blue!10, colframe=blue!40!black, title=Doubt]
}{
  \end{tcolorbox}
}

\newenvironment{deduction}{
  \begin{tcolorbox}[colback=green!10, colframe=green!40!black, title=Deduction]
}{
  \end{tcolorbox}
}

\title{How to control a stteper motor}
\author{José Agustín}
\date{\today}
\begin{document}
\maketitle 
\tableofcontents 

\chapter{Introduction}
In this document, I will attemp to explain all the effects involved in the control
of a stepper motor and how to build a custom driver with the design of a custom PCB.
The objective is to create the main component for providing feedback to a Steering Wheel.
This Documents it's somewhat like a guide on how to create a driver(PCB), covering:
\begin{itemize}
\item Effects involved in stepper motor
\item Stepper motor characteristics
\item Electronics components for motor control
\item Market solutions analysis for PCB design
\item PCB design Guidelines for stepper motor control
\item Selection of electronics components
\item Simulation with Ngspice
\end{itemize}
All of this topics will be illustrated with a real-world example.\\
\textbf{Note:} this is the first version of this document, so there may be 
some errors for which i am not responsible

\chapter{Effects involved in stepper motor}
\section{What Is a Stepper Motor?}
A stepper motor is an electrical motor, and like any electric motor they have coils, the main 
difference with others motors it's his construction, with his construction it have some interesting
capabilities, like rotate in discrete \textbf{steps}, rotating in fixed angles such 0.8\degree or 
1.9\degree, althought the exact step size depends in themanufacture.\\
To move(rotate) a stepper motor, a special driver is necessary that provides the correct current, voltage and
the order to energize the coils.
\section{What are the physical and electrical pfenomemals involved?}
\subsection{Torque and load effects}
\subsection{Magnetic flux path}
\subsection{Resonance phenomena}
\subsection{Step accuracy and resolution}
\subsection{Heat gereration and cooling}
\subsection{Electromagnetic interference (EMI)}
\subsection{Electromagnetic coil behavior}
\subsection{Inductance and impedance}
\subsection{Back EMF(Electromotive Force)}

\chapter{Stepper motor characteristics}
\section{How its the construction of a common Stepper Motor?}

\chapter{Notes}
In this chapter I will keep adding notes based in some references to compare and support the information 
showed here.
\section{AN907|Microchip Technology Inc}

\subsection*{A stepper motor is load independent as long as the load does not exceed the torque rating}
\hrule
\vspace{4pt}
That is correct because when you apply a small force with your fingertips to the shaft of
a stepper motor, you can't notice the a decrease in force or velocity.\\
But if you attempt to do the same with a typical brush motor you can find out that it will
start to decrease its velocity and force, of course, if doesn't have a feedback circuit.

 
\subsection*{Types of Stepper motors}
\begin{itemize}
	\item Permanent magnet: Have a magnetized rotor
	\item Variable reluctance: Have soft-iron rotor
	\item Hybrid: Combine aspect of both
\end{itemize}

\begin{doubt}
	\textbf{What is a Soft-iron rotor?}\\
	Its material composition is soft-iron, which provides certain capabilities that are different
	from those of a permanent magnet. These capabilities will be discussed in another note.
\end{doubt}

\subsection*{What are the parts of an electric motor}
\begin{itemize}
  \item Stator: Its the stationary part of the stepping motor that holds the multiples windings
  \item Rotor:
\end{itemize}



\subsection*{Variable Reluctance Motor}
They hace three to five windings connected to a common terminal.\\
When a winding is energized, start an atraction that is caused by the magnetic flux path generated around de coil
and the rotor, the rotor expiriences a torque and move the rotor in line with the energized coils, \textbf{minimizing
the flux path.}\\

Continuos clockwise motion is achieved by sequentially energizing and de-energizing windings around the stator.\\

The number of poles can be greater by adding windings, for example, moving to 4 or 5 windings, \textbf{but for small steps angles 
the usual solution is to use tothed pole pieces working against a toothed rotor.}

\subsection*{Unipolar Motor}
Unipolar stepping motor are \textbf{composed of two windings, each one with a center tap}, the center taps are either brought outside
of the motor as two separed wires or connected to each other internally and brought outside the motor as one wire, the center tap wire(s)
is tied to a power supply and the ends of the coils are alternately grounded. 

These motors operate by attracting the north or south poles of the permanent magnetized rotor to the stator poles. 

The direction of the current through the stator windings determine which rotor poles will be atracted to which stator poles.

Physically, the half of the windings are wound parallel to one another. Therefore, \textbf{one winding acts as either a north or 
south pole depending of which half is powered}.\\

\textbf{Only half of each winding is powered at a time}

\subsection*{Bipolar Motor}
Bipolar stepping motors are compose of two windings and have four wires but don't have center taps.\\
The advantage is that current runs through an entire winding at a time, resulting in more torque compared with 
an unipolar motor as the same size, the draw back is that more complex control circuitry is required.\\

Current flow in the winding of a bipolar motor is \textbf{bidirectional}, this requires changing the polarity of
each end of the winding, to change the polarity of the ends is used a control circuit know as an \textbf{H-bridge}

\textbf{Every Bipolar motor has two windings, therefore, two H-bridge control circuit are needed for each motor}

\subsection*{Bifilar Motor}
\textbf{Means two threaded}, This motors are indentical in rotor and stator to bipolar motors with one exeption, each 
winding is made up of two wires wound in parallel to each other. 

Bifilar motors are driven as either unipolar or bipolar motors

\subsection*{Hybrid Motors}
Share the principles of both permanent and variable reluctance motor steeping motor. 

The rotor for a hybrid stepping motor is multitoothed, like the variable reluctance motor, and contains and axially
magnetized concetric magnet around its shaft.

\section*{Variable Reluctance Motor vs Permanent magnet or hybrid}

\subsection*{Variable Reluctance Drawbacks}
\begin{itemize}
  \item Are generally noisy
  \item Complex current limiting control is required to achieve high speeds.
  \item Microstepping is not generally applicable 
\end{itemize}

\subsection*{Variable Reluctance advantages}
\begin{itemize}
  \item The drop in torque with speed is less pronounced
  \item Simple design
  \item More robust than permanent magnet motors
\end{itemize}

\subsection*{Permanent Magnetic and hybrid Drawbacks}
\begin{itemize}
  \item Cog when they are turned by hand while not powered 
\end{itemize}

\subsection*{Permanent Magnetic and hybrid advantages}
\begin{itemize}
  \item Are generally preferred where noise or vibration are issues
  \item Can be microstepped
\end{itemize}

\section*{Unipolar versus Bipolar}
Bipolar motor have approximately 30\% more torque than an equivalent Unipolar motor with the same size,
but require more complex control circuitry than Unipolar motor. 

\section*{Torque}
Stepper motors have different types of rated torque. These are:
\begin{itemize}
  \item \textbf{Holding torque:} The torque required to rotate the
        motor’s shaft while the windings are energized.
  \item \textbf{Pull-in torque:} The torque against which a motor
        can accelerate from a standing start without
        missing any steps, when driven at a constant
        stepping rate.
  \item \textbf{Pull-out torque:} The load a motor can move
        when at operating speed.
  \item \textbf{Detent torque:} The torque required to rotate the
        motor’s shaft while the windings are not
        energized.
\end{itemize}
\end{document}
  \State 
\EndWhile
