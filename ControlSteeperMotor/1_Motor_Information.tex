\documentclass{report} 
\usepackage[english]{babel}
\usepackage[utf8]{inputenc}
\usepackage{gensymb}
\usepackage{tcolorbox}

\newenvironment{deductionR}{
  \begin{tcolorbox}[colback=orange!10, colframe=orange!40!black, title=Deduction result]
}{
  \end{tcolorbox}
}

\newenvironment{doubt}{
  \begin{tcolorbox}[colback=blue!10, colframe=blue!40!black, title=Doubt]
}{
  \end{tcolorbox}
}

\newenvironment{deduction}{
  \begin{tcolorbox}[colback=green!10, colframe=green!40!black, title=Deduction]
}{
  \end{tcolorbox}
}

\title{How to control a stteper motor}
\author{José Agustín}
\date{\today}
\begin{document}
\maketitle 
\tableofcontents 

\chapter{Introduction}
In this document, I will attemp to explain all the effects involved in the control
of a stepper motor and how to build a custom driver with the design of a custom PCB.
The objective is to create the main component for providing feedback to a Steering Wheel.
This Documents it's somewhat like a guide on how to create a driver(PCB), covering:
\begin{itemize}
\item Effects involved in stepper motor
\item Stepper motor characteristics
\item Electronics components for motor control
\item Market solutions analysis for PCB design
\item PCB design Guidelines for stepper motor control
\item Selection of electronics components
\item Simulation with Ngspice
\end{itemize}
All of this topics will be illustrated with a real-world example.\\
\textbf{Note:} this is the first version of this document, so there may be 
some errors for which i am not responsible

\chapter{Effects involved in stepper motor}
\section{What Is a Stepper Motor?}
A stepper motor is an electrical motor, and like any electric motor they have coils, the main 
difference with others motors it's his construction, with his construction it have some interesting
capabilities, like rotate in discrete \textbf{steps}, rotating in fixed angles such 0.8\degree or 
1.9\degree, althought the exact step size depends in themanufacture.\\
To move(rotate) a stepper motor, a special driver is necessary that provides the correct current, voltage and
the order to energize the coils.
\section{What are the physical and electrical pfenomemals involved?}
\subsection{Torque and load effects}
\subsection{Magnetic flux path}
\subsection{Resonance phenomena}
\subsection{Step accuracy and resolution}
\subsection{Heat gereration and cooling}
\subsection{Electromagnetic interference (EMI)}
\subsection{Electromagnetic coil behavior}
\subsection{Inductance and impedance}
\subsection{Back EMF(Electromotive Force)}

\chapter{Stepper motor characteristics}
\section{How its the construction of a common Stepper Motor?}

\chapter{Notes}
In this chapter I will keep adding notes based in some references to compare and support the information 
showed here.
\section{AN907|Microchip Technology Inc}

\subsection*{A stepper motor is load independent as long as the load does not exceed the torque rating}
\hrule
\vspace{4pt}
That is correct because when you apply a small force with your fingertips to the shaft of
a stepper motor, you can't notice the a decrease in force or velocity.\\
But if you attempt to do the same with a typical brush motor you can find out that it will
start to decrease its velocity and force, of course, if doesn't have a feedback circuit.

 
\subsection*{Types of Stepper motors}
\begin{itemize}
	\item Permanent magnet: Have a magnetized rotor
	\item Variable reluctance: Have soft-iron rotor
	\item Hybrid: Combine aspect of both
\end{itemize}

\begin{doubt}
	\textbf{What is a Soft-iron rotor?}\\
	Its material composition is soft-iron, which provides certain capabilities that are different
	from those of a permanent magnet. These capabilities will be discussed in another note.
\end{doubt}

\subsection*{Variable Reluctance Motor}
They hace three to five windings connected to a common terminal


\end{document}
